\chapter{Ohjelmointiparadigmat} \label{Ohjelmointiparadigmat}
Ohjelmointiparadigmat luokittelevat ohjelmointikieliä sen perusteella, miten kieli on suunniteltu mallintamaan ongelmia ja minkälaisia mekanismeja kieli tarjoaa näiden ongelmien ratkaisuun. Yhdessä ohjelmointikielessä voi olla vaikutteita useammasta paradigmasta. Tällaista kieltä kutsutaan \textit{moniparadigmaiseksi} kieleksi. Ohjelmointiparadigmat voi jakaa karkeasti kahteen yläluokkaan: \textit{imperatiivisiin} ja \textit{deklaratiivisiin}.
\cite[Luku 6]{principlesAndParadigms}

Imperatiiviset kielet keskittyvät ohjelmointiongelman ratkaisuun määrittelemällä \textit{miten} tietokoneen tulisi ratkaista laskettava ongelma. Ohjelmat siis rakentuvat peräkkäisistä käskyistä, jotka muokkaavat ohjelman tilaa ja näin ratkaisevat ongelman. Imperatiivisista kielistä puhutaan matalan tason kielinä, sillä ongelman ratkaisu mallinnetaan niissä tietokoneen näkökulmasta.
\cite[Luku 1]{programmingLanguagePragmatics} 

Deklaratiiviset kielet ratkaisevat ongelman vastaamalla kysymykseen \textit{mitä} tietokoneen tulisi tehdä ongelman ratkaisemiseksi. Käytännössä tämä tarkoittaa ongelman ratkaisun kuvailemista ilman toteutuksen yksityiskohtiin uppoutumista. Deklaratiiviset mielletään yleensä korkean tason kielinä, sillä ne mallintavat ongelmanratkaisua ohjelmoijan näkökulmasta.
\cite[Luku 1]{programmingLanguagePragmatics}


\section{Olioparadigma}
Olio-ohjelmoinnin perusajatus on kuvata ongelmaa olioiden avulla, jotka kukin kuvaavat jotakin ongelma-alueen käsitettä. Olioiden tarkoitus on kapseloida kuvaamansa käsitteen tieto ja tila sisäänsä sekä tarjota operaatioita kapseloidun tiedon muokkaamiseen ja tarkasteluun. Olioparadigma on osa imperatiivisia paradigmoja, sillä ongelmanratkaisu tapahtuu peräkkäisillä komennoilla, jotka muuttavat olioiden ja samalla koko ohjelman tilaa kohti ratkaistua ongelmaa. \cite[Luku 1]{programmingLanguagePragmatics}
\cite[Luku 10]{principlesAndParadigms}


\section{Funktionaalinen paradigma}
Funktionaaliset ohjelmat rakentuvat yksittäisistä funktioista, joita yhdistämällä luodaan yhä isompia funktiota. Näillä funktioilla ei ole tilaa, ja ne ovat puhtaita eli eivät aiheuta \textit{sivuvaikutuksia}. Sivuvaikutuksia voivat olla esimerkiksi muuttujan arvon muuttaminen, poikkeuksen nostaminen, tiedoston lukeminen tai kirjoittaminen, pyyntö tietokantaan tai verkon ylitse sekä näytölle piirtäminen. Tilaton ja puhdas funktio palauttaa tilanteesta riippumatta tietyllä syötteellä aina saman arvon. \cite[Luku 1]{functionalProgrammingInScala}

Ohjelmointikieli ilman sivuvaikutuksia olisi käytännössä hyödytön, joten funktionaaliset ohjelmointikielet tarjoavat erilaisia mekanismeja sivuvaikutusten hallintaan.  Funktionaalisen ohjelmoinnin katsotaan yleensä kuuluvan deklaratiiviseen paradigmaan. \cite[Luku 11]{principlesAndParadigms}


\section{Paradigmoille tyypillisiä ominaisuuksia}
Lauseke on ohjelmointikielen rakenne, jonka suoritus tuottaa arvon. Lauseke voi koostua joko arvosta tai operaattorista, jota on sovellettu yhteen tai useampaan lausekkeeseen. Esimerkiksi \code{3+2} on numeerinen lauseke, jossa on yksi operaattori \code{+}, jota sovelletaan kahteen numeeriseen arvoon \code{3} ja \code{2}. Lausekkeen arvo voidaan sijoittaa muuttujaan, antaa parametriksi funktioon tai sen arvo voidaan palauttaa funktiosta. Lauseke on minkä tahansa ohjelmointikielen perusrakenne, eli lausekkeita on funktionaalisissa sekä oliokielissä.
\cite[Luku 6]{principlesAndParadigms}

Funktionaalisessa ja olioparadigmassa molemmissa on muuttujia, mutta niitä käsitellään eri tavoilla. Funktionaalisissa kielissä muuttujaan sijoitettua arvoa ei voi muuttaa alustuksen jälkeen, kun taas imperatiivissa kielissä muuttujan arvoa voi yleensä muuttaa. Sama pätee tietorakenteisiin: imperatiivissa kielissä niiden muuttaminen alustamisen jälkeen on sallittua, funktionaalisissa ei. \cite[Luku 3]{functionalProgrammingInScala} Funktionaalisissa kielissä funktioita kohdellaan kuin mitä tahansa muitakin arvoja: niitä voi sijoittaa muuttujiin, antaa parametrina toiseen funktioon tai käyttää funktion palautusarvona. Oliokielissä tämä ei aina ole mahdollista.
\cite[Luku 6]{principlesAndParadigms}

Komento on ohjelmointikielen rakenne, jonka suoritus ei aina tuota arvoa ja saattaa aiheuttaa sivuvaikutuksia. Komennot saattavat esimerkiksi muuttaa olioiden ja muuttujien tilaa, lukea käyttäjän tuottamaa syötettä tai nostaa poikkeuksen. Komentojen suorittamisella saattaa olla sivuvaikutuksia, joten niiden suoritusjärjestys on merkityksellinen. Imperatiivisiin kieliin komennot kuuluvat oleellisesti, mutta funktionaalisiin eivät.

Imperatiivisissa kielissä toisto toteutetaan silmukoilla. Silmukkaa suoritetaan ennalta määritelty määrä tai kunnes silmukan suoritusta säätelevä ehtolause muuttuu epätodeksi. Ehtolauseen arvo määräytyy muuttujan arvon mukaan, jota muutetaan silmukan suorituksen aikana. Funktionaalisissa kielissä muuttujien arvoa ei voi muuttaa, ja siitä syystä silmukat olisivat hyödyttömiä. Silmukoiden sijaan käytetään rekursiota, eli funktio kutsuu itseään, kunnes funktioon määritelty perustapaus saavutetaan.
\cite[Luku 6 ja 11]{principlesAndParadigms}

Kummankin paradigman ominaisuuksissa on omat ongelmakohtansa. Imperatiivisia ohjelmia on tavallisesti vaikea säikeistää, sillä jos useat säikeet käsittelevät samaa oliota samanaikaisesti, saattaa olion tila olla odottamaton. Funktionaalisessa ohjelmassa samaa ongelmaa ei ole, koska luomisen jälkeen olion tila ei muutu, ja sitä voi turvallisesti käyttää useampi säie. 
\cite[Luku 6]{prorgrammingInScala3rd}

Toiston toteuttaminen rekursion avulla toimii hyvin, kunhan rekursio ei ajaudu liian syväksi. Rekursiivinen funktiokutsu kasvattaa kutsupinoa, kunnes pino vuotaa yli ja ohjelma kaatuu. Tämä tarkoittaa, että käsiteltävän datan määrä ei voi olla suuri tai kääntäjän pitää pystyä tekemään häntäoptimointia. Häntäoptimointi on tapa, jolla rekursiivinen kutsu voidaan muuttaa silmukaksi, jos se on funktion viimeinen operaatio.
\cite[Luku 8]{prorgrammingInScala3rd} 
