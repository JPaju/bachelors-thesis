
\keywords{suorituskyky, funktionaalinen ohjelmointi, ohjelmointiparadigma, Scala}

\begin{abstract}
Funktionaalisesta ohjelmoinnista lainattuja ominaisuuksia lisätään jatkuvasti perinteisiin imperatiivisiin ohjelmointikieliin. Yksi funktionaalisen ohjelmoinnin keskeisimmistä käsitteistä on muuttumattomat arvot. Muuttumattomien arvojen ja tietorakenteiden käyttö saattaa lisätä kopioimisen tarvetta ja samalla heikentää ohjelman suorituskykyä.

Tutkielman tarkoituksena on perehtyä funktionaalisten ohjelmien suorituskykyyn vaikuttaviin seikkoihin sekä verrata funktionaalisen paradigman suorituskykyä imperatiiviseen paradigmaan. Vertailut tehdään Scala-kielellä, sillä se tukee kumpaakin edellä mainittua paradigmaa. Tutkielmassa keskitytään järjestettyjen muuttumattomien ja muuttuvien kokoelmien suorituskyvyn vertailuun. Suorituskykyä tutkitaan useamman eri mittauksen pohjalta, ja mittauksien tuloksia vertaillaan toisiinsa.

Mittauksissa ei havaittu säännönmukaisia eroja muuttumattomien ja muuttuvien kokoelmien suorituskyvyssä, vaikka muuttuvat kokoelmat olivat joissain mittauksissa hieman muuttumattomia suorituskykyisempiä. Suurin vaikutus suorituskykyyn on tarkoituksenmukaisen tietorakenteen valitsemisella riippumatta onko kyseessä muuttuva vai muuttumaton kokoelma.

Tutkielman pohjalta ei voida tehdä johtopäätöksiä funktionaalisen paradigman kokonaisvaltaisesta vaikutuksesta ohjelman suorituskykyyn. Kokonaisvaltaisten vaikutusten arvioimiseksi tulisi tutkia myös muita funktionaalisen paradigman keskeisiä käsitteitä, kuten sulkeumia, hahmontunnistusta ja rekursiota.
\end{abstract}
