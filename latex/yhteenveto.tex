\chapter{Yhteenveto} \label{Yhteenveto}
Tutkielmassa pyrittiin selvittämään funktionaalisen ohjelmointiparadigman vaikutuksia suorituskykyyn verrattuna imperatiiviseen paradigmaan. Vaikutuksia suorituskykyyn pyrittiin tarkastelemaan vertailemalla järjestettyjen muuttuvien ja muuttumattomien kokoelmaluokkien suorituskykyjä toisiinsa. Suorituskykyä tarkasteltiin tutustumalla lähteenä käytettyihin suorituskykymittauksiin. Tarkasteluun valittiin kullekin paradigmalle tyypillisiä kokoelmaluokkia. Johtopäätöksiä kokoelmaluokkien suorituskyvystä tehtiin vertailemalla lähteiden mittauksien tuloksia. 

Selviä eroja muuttuvien ja muuttumattomien kokoelmien suorituskyvylle ei havaittu, vaikka muuttuvat kokoelmat näyttäisivätkin suoriutuvan useissa operaatioissa hieman muuttumattomia paremmin. Suorituskyvyn kannalta tärkein tekijä oli kuitenkin kuhunkin käyttötarkoitukseen sopivan tietorakenteen valinta. Tarkempien säännönmukaisuuksien selvittämiseksi tulisi mittauksia tehdä useamman kokoelmaluokan, operaation ja skenaarion yhdistelmistä. Myös järjestämättömien kokoelmien, kuten joukkojen ja assosiaatiotaulujen, suorituskykyä tulisi vertailla.

Kokonaisvaltaista johtopäätöstä funktionaalisen paradigman suorituskykyvaikutuksista imperatiiviseen paradigmaan verrattuna ei tämän tutkielman tulosten perusteella voi tehdä. Jatkotutkimuskohteena kokoelmien osalta voisivat olla laiskat ja rinnakkaiset kokoelmat. Muita tutkielman tekemisen aikana esiin nousseita jatkotutkimuksen aiheita ovat hahmontunnistus, sulkeumat, funktioiden osittainen soveltaminen sekä silmukoiden ja rekursion suorituskyvyn vertailu.
