\chapter{Johdanto} \label{Johdanto}
Funktionaalinen ohjelmointi kasvattaa suosiotaan jatkuvasti. Useisiin yleiskäyttöisiin ja alun perin imperatiivisiin ohjelmointikieliin on lisätty ominaisuuksia funktionaalisesta ohjelmoinnista. Esimerkiksi suositut oliokielet Java, Python ja C++ ovat kaikki lainanneet funktionaalisesta ohjelmoinnista anonyymit sekä korkeamman asteen funktiot.

Funktionaalisen paradigman keskeisiä käsitteitä ovat muuttumattomat arvot ja tietorakenteet. Muuttumattomien arvojen käyttäminen usein lisää kopioimisen tarvetta verrattuna ohjelmiin, joissa käytetään muuttuvia arvoja. Kopioimisen seurauksena muistia pitää varata ja vapauttaa useammin kuin muuttuvia arvoja käytettäessä. Tämä herättää kysymyksen funktionaalisten ohjelmien suorituskyvystä verrattuna imperatiivisiin ohjelmiin.

Tutkielmassa pyritään saamaan vastaus siihen, millaisia vaikutuksia suorituskykyyn funktionaalisen paradigman käytöllä on verrattuna olioparadigmaan. Suorituskykyvertailut keskittyvät Scala-ohjelmointikieleen, sillä se tukee kumpaakin paradigmaa, jolloin suorituskyvyn vertailu paradigmojen välillä on mielekästä ja suoraviivaista. Tarkastelu kohdistuu muuttumattomiin tietorakenteisiin ja niihin liittyviin algoritmeihin. Tutkielma on suoritettu perehtymällä aihetta käsittelevään kirjallisuuteen ja suorituskykymittauksiin.

Luvussa \ref{Ohjelmointiparadigmat} esitellään molemmat vertailun kohteena olevat paradigmat. Luvussa \ref{Scala} esitellään Scala-kielen rakenteet, ja miten ne tukevat kumpaakin paradigmaa. Luvussa \ref{Kokoelmat} tarkastellaan Scalan standardikirjaston muuttumattomien kokoelmien suorituskykyä ja verrataan sitä muuttuvien kokoelmien suorituskykyyn. Viimeisenä luvussa \ref{Yhteenveto} esitellään johtopäätökset ja kootaan tutkielman tulokset.
