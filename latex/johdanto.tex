\chapter{Johdanto} \label{Johdanto}
Funktionaalinen ohjelmointi kasvattaa suosiotaan jatkuvasti. Useisiin yleiskäyttöisiin ja alunperin imperatiivisiin ohjelmointikieliin on lisätty ominaisuuksia funktionaalisesta ohjelmoinnista. Esimerkiksi suositut oliokielet Java, Python ja C++ ovat kaikki lainanneet funktionaalisesta ohjelmoinnista anonyymit sekä korkeamman asteen funktiot.

Funktionaalisen paradigman keskeisistä käsitteitä ovat muuttumattomat arvot ja tietorakenteet. Muuttumattomien arvojen käyttäminen usein lisää kopioimisen tarvetta verrattuna ohjelmiin, joissa käytetään muuttuvia arvoja. Kopioimisen seurauksena muistia pitää varata ja vapauttaa useammin kuin muuttuvia arvoja käytettäessä. Tämä herättää kysymyksen funktionaalisten ohjelmien suorituskyvystä verrattuna imperatiivisiin ohjelmiin.

Tutkielmassa perehdytään tarkastelemaan millaisia vaikutuksia funktionaalisella paradigmalla on suorituskykyyn verrattuna olioparadigmaan. Tutkielman suorituskykyvertailut keskittyvät Scala-ohjelmointikieleen, sillä se on suunniteltu tukemaan sekä funktionaalista että olioparadigmaa, jolloin suorituskyvyn vertailu näiden kahden paradigman välillä on mielekästä ja suoraviivaista. Tarkastelu kohdistuu erityisesti muuttumattomiin kokoelmiin. \todo{Lisää toinen tutkittava aihe} Tutkielma on suoritettu perehtymällä aihetta käsittelevään kirjallisuuteen.

Luvussa \ref{Ohjelmointiparadigmat} esitellään molemmat vertailun kohteena olevat paradigmat. Luvussa \ref{Scala} esitellään Scala-kielen rakenteet, ja miten ne tukevat kumpaakin paradigmaa. Luvussa \ref{Kokoelmat} tarkastellaan Scalan standardikirjaston muuttumattomien kokoelmien suorituskykyä ja verrataan sitä muuttuvien kokoelmien suorituskykyyn. \todo{Varmista että suorituskykyä oikeasti verrataan muuttuviin kokoelmiiin} Viimeisenä luvussa \ref{Yhteenveto} esitellään johtopäätökset ja kootaan tutkielman tulokset.

\todo{Lisää esittely toisesta tutkittavasta aiheesta}
